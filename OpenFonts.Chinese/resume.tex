%%%%%%%%%%%%%%%%%%%%%%%%%%%%%%%%%%%%%%%
% Deedy - One Page Two Column Resume
% LaTeX Template
% Version 1.2 (16/9/2014)
%
% Original author:
% Debarghya Das (http://debarghyadas.com)
%
% Original repository:
% https://github.com/deedydas/Deedy-Resume
%
% IMPORTANT: THIS TEMPLATE NEEDS TO BE COMPILED WITH XeLaTeX
%
% This template uses several fonts not included with Windows/Linux by
% default. If you get compilation errors saying a font is missing, find the line
% on which the font is used and either change it to a font included with your
% operating system or comment the line out to use the default font.
% 
%%%%%%%%%%%%%%%%%%%%%%%%%%%%%%%%%%%%%%
% 
% TODO:
% 1. Integrate biber/bibtex for article citation under publications.
% 2. Figure out a smoother way for the document to flow onto the next page.
% 3. Add styling information for a "Projects/Hacks" section.
% 4. Add location/address information
% 5. Merge OpenFont and MacFonts as a single sty with options.
% 
%%%%%%%%%%%%%%%%%%%%%%%%%%%%%%%%%%%%%%
%
% CHANGELOG:
% v1.1:
% 1. Fixed several compilation bugs with \renewcommand
% 2. Got Open-source fonts (Windows/Linux support)
% 3. Added Last Updated
% 4. Move Title styling into .sty
% 5. Commented .sty file.
%
%%%%%%%%%%%%%%%%%%%%%%%%%%%%%%%%%%%%%%%
%
% Known Issues:
% 1. Overflows onto second page if any column's contents are more than the
% vertical limit
% 2. Hacky space on the first bullet point on the second column.
%
%%%%%%%%%%%%%%%%%%%%%%%%%%%%%%%%%%%%%%


\documentclass[]{deedy-resume-openfont}
\usepackage{fancyhdr}
    
\pagestyle{fancy}
\fancyhf{}
    
\begin{document}

%%%%%%%%%%%%%%%%%%%%%%%%%%%%%%%%%%%%%%
%
%     LAST UPDATED DATE
%
%%%%%%%%%%%%%%%%%%%%%%%%%%%%%%%%%%%%%%
\lastupdated

%%%%%%%%%%%%%%%%%%%%%%%%%%%%%%%%%%%%%%
%
%     TITLE NAME
%
%%%%%%%%%%%%%%%%%%%%%%%%%%%%%%%%%%%%%%
\namesection{马}{昀哲}{ \urlstyle{same}\href{mailto:i@yunze.me}{i@yunze.me} | +86-13720517478
}

%%%%%%%%%%%%%%%%%%%%%%%%%%%%%%%%%%%%%%
%
%     COLUMN ONE
%
%%%%%%%%%%%%%%%%%%%%%%%%%%%%%%%%%%%%%%

\begin{minipage}[t]{0.25\textwidth} 

%%%%%%%%%%%%%%%%%%%%%%%%%%%%%%%%%%%%%%
%     EDUCATION
%%%%%%%%%%%%%%%%%%%%%%%%%%%%%%%%%%%%%%

\section{教育经历} 
\sectionsep

\subsection{东京工业大学}
\descript{硕士学位,计算机科学专业}
\location{GPA: 3.7 / 4.0}
\location{东京, 日本}
\location{2020.04-2022.03 (预计)}
\sectionsep

\subsection{浙江大学}
\descript{学士学位,计算机科学专业}
\location{GPA: 3.5 / 4.0}
\location{2013.09-2017.07}
\sectionsep

%%%%%%%%%%%%%%%%%%%%%%%%%%%%%%%%%%%%%%
%     LINKS
%%%%%%%%%%%%%%%%%%%%%%%%%%%%%%%%%%%%%%

\section{链接}
\sectionsep
Github:// \href{https://github.com/myzWILLmake}{\bf myzWILLmake} \\
% LinkedIn://  \href{https://www.linkedin.com/in/qiyuqi}{\bf qiyuqi} \\

%%%%%%%%%%%%%%%%%%%%%%%%%%%%%%%%%%%%%%
%     COURSEWORK
%%%%%%%%%%%%%%%%%%%%%%%%%%%%%%%%%%%%%%

\section{修读课程}
\subsection{硕士}
分布式算法 \\
计算机网络 \\
密码学 \\
虚拟现实技术 \\
英语口语 \\ 
\sectionsep

\subsection{学士}
数据结构与算法 \\
操作系统原理 \\ 
数据库原理 \\
Python 开发技术 \\
信息系统安全 \\

%%%%%%%%%%%%%%%%%%%%%%%%%%%%%%%%%%%%%%
%     SKILLS
%%%%%%%%%%%%%%%%%%%%%%%%%%%%%%%%%%%%%%

\section{技能}
\sectionsep
\subsection{编程}
\location{超过 5000 行}
Go \textbullet{} C++ \textbullet{} Lua \textbullet{} Python \\
\location{1000 - 5000 行}
C\# \textbullet{} Javascript \textbullet{} C \textbullet{} HTML \\
\location{低于 1000 行}
SQL \textbullet{} Shell \textbullet{} Rust \\ 
\sectionsep

\subsection{工具 | 框架}
\location{一般}
Git \textbullet{} Linux \textbullet{} Redis \textbullet{} Unity \textbullet{} Markdown \\
\location{了解}
Docker  \textbullet{} SVN \textbullet{} MongoDB \textbullet{} Regex \textbullet{} Vue.js \\
\sectionsep

%%%%%%%%%%%%%%%%%%%%%%%%%%%%%%%%%%%%%%
%
%     COLUMN TWO
%
%%%%%%%%%%%%%%%%%%%%%%%%%%%%%%%%%%%%%%

\end{minipage} 
\hfill
\begin{minipage}[t]{0.74\textwidth} 

%%%%%%%%%%%%%%%%%%%%%%%%%%%%%%%%%%%%%%
%     EXPERIENCE
%%%%%%%%%%%%%%%%%%%%%%%%%%%%%%%%%%%%%%

\section{工作经历}

\runsubsection{网易(杭州)网络有限公司}
\descript{游戏后端开发工程师}
\location{2018.11 - 2019.04 | 杭州, 中国}
\vspace{\topsep}
\begin{tightemize}
    \item 开发双人对战卡牌手机游戏的服务端,Lua作为脚本运行层,c++作为服务器引擎。
    \item 完成了服务端逻辑的开发,包括成就系统,GM管理系统,邮件系统和事件系统。
\end{tightemize}
\sectionsep

\runsubsection{网易(杭州)网络有限公司}
\descript{游戏后端开发工程师}
\location{2017.07 - 2018.11 | 杭州, 中国}
\vspace{\topsep}
\begin{tightemize}
    \item 开发多人实时对战手机游戏的服务端,Lua作为脚本运行层,c++作为服务器引擎。
    \item 设计并实现了服务器的消息传递框架,并通过了压力测试和线上正式运行。
    \item 实现了服务器机器人测试框架,为压力测试提供了技术支持。
\end{tightemize}
\sectionsep

\runsubsection{网易(杭州)网络有限公司}
\descript{游戏开发工程师(实习)}
\location{2016.08 - 2017.05 | 杭州, 中国}
\vspace{\topsep}
\begin{tightemize}
    \item 参与项目初期游戏原型demo的开发,并通过了内部审核成立了正式项目。
    \item 设计并实现了游戏观战系统,并通过了线上正式运行。
\end{tightemize}
\sectionsep

%%%%%%%%%%%%%%%%%%%%%%%%%%%%%%%%%%%%%%
%     RESEARCH
%%%%%%%%%%%%%%%%%%%%%%%%%%%%%%%%%%%%%%

\section{项目}

\runsubsection{{\bf Consensus-go}}
\descript{作者}
\location{2020.4 - 2020.9}
\begin{tightemize}
    \item 基于go语言,实现了包括paxos、raft、pbft、hotstuff等共识算法。
    \item 为实验室之后针对分布式共识算法研究提供了工程上的验证支持。
    \end{tightemize}
\sectionsep

\runsubsection{{\bf 游戏观战系统}}
\descript{作者}
\location{2017.02 - 2017.05}
\begin{tightemize}
    \item 该项目使用Unity作为客户端,Python作为直播流服务器。
    \item 将直播和回放数据和服务器解耦,降低了服务端处理压力,并可以进行动态部署。
    \item 项目框架在工作室之后的其他对战游戏中被采用,并通过了线上正式运行。
    \end{tightemize}
\sectionsep


%%%%%%%%%%%%%%%%%%%%%%%%%%%%%%%%%%%%%%
%     OPEN SOURCE
%%%%%%%%%%%%%%%%%%%%%%%%%%%%%%%%%%%%%%

\section{开源贡献}

\begin{tabular}{ll}
\href{https://github.com/myzWILLmake/pbft-go}{\bf hotstuff-go} & 使用go语言实现的PBFT分布式共识算法 \\
\href{https://github.com/myzWILLmake/bgmgo}{\bf bgmgo} & 使用go语言实现的节目订阅的命令行工具 \\
\href{https://github.com/poooi/poi}{\bf poi} & 使用Electron和React实现的跨平台网页游戏专用浏览器 \\
\end{tabular}
\sectionsep

%%%%%%%%%%%%%%%%%%%%%%%%%%%%%%%%%%%%%%
%     AWARDS
%%%%%%%%%%%%%%%%%%%%%%%%%%%%%%%%%%%%%%

\section{所获奖项}

\begin{tabular}{rll}
2020      & 奖学金 & 日本政府面向交换生的JASSO奖学金 \\
2016	     & 三等奖  & 浙江大学学业奖学金 \\
2012	     & 一等奖  & 全国信息学奥林匹克联赛(NOIP)\\
\end{tabular}
\sectionsep


%%%%%%%%%%%%%%%%%%%%%%%%%%%%%%%%%%%%%%
%     LANGUAGE
%%%%%%%%%%%%%%%%%%%%%%%%%%%%%%%%%%%%%%

\section{语言} 

\begin{tabular}{rll}

英语   & 流畅 & 托业: 905pt \\
日语	     & 日常会话 &   \\
中文	     & 母语  &  \\
\end{tabular}

%%%%%%%%%%%%%%%%%%%%%%%%%%%%%%%%%%%%%%
%     PUBLICATIONS
%%%%%%%%%%%%%%%%%%%%%%%%%%%%%%%%%%%%%%

% \section{Publications} 
% \renewcommand\refname{\vskip -1.5cm} % Couldn't get this working from the .cls file
% \bibliographystyle{abbrv}
% \bibliography{publications}
% \nocite{*}

\end{minipage} 
\end{document}  \documentclass[]{article}
